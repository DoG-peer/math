\documentclass[uplatex]{jsarticle}

\usepackage{amsmath,amsthm,amssymb,amscd}
\usepackage{ascmac}
\usepackage[all]{xy}
\usepackage{enumerate}
\usepackage{graphicx}
\usepackage{mathrsfs}

\theoremstyle{definition}
\numberwithin{equation}{section}
\newtheorem{prob}[section]{問題}

\newcommand{\midbar}{\ \middle| \ }

\newcommand{\mf}[1]{{\mathfrak{#1}}}
\newcommand{\bm}[1]{{\mathbf{#1}}}
\newcommand{\bb}[1]{{\mathbb{#1}}}
\newcommand{\mca}[1]{{\mathcal{#1}}}
\newcommand{\msc}[1]{{\mathscr{#1}}}
\newcommand{\mm}[1]{\mathop{\mathrm{#1}}\nolimits}

\newcommand{\R}{\mathbb{R}}
\newcommand{\C}{\mathbb{C}}
\newcommand{\Z}{\mathbb{Z}}
\newcommand{\Q}{\mathbb{Q}}

\newcommand{\Zn}{\mathbb{Z}/n\mathbb{Z}}
\newcommand{\Zp}{\mathbb{Z}/p\mathbb{Z}}

\begin{document}

\begin{prob}
  それぞれに対して、条件Aが条件Bの、必要条件、十分条件、必要十分条件、もしくはいずれでもない、のうちどれになっているか答えよ。(必要十分条件のときは、必要十分条件と答えよ。)
  \begin{enumerate}
    \item $x$,$y$は整数とする。
      \begin{enumerate}
        \item[(A)] $(x-y)^2 < 1$.
        \item[(B)] $x=y$.
      \end{enumerate}
    \item 三角形$\triangle{ABC}$, $\triangle{DEF}$を考える。
      \begin{enumerate}
        \item[(A)] $\triangle{ABC}\equiv \triangle{DEF}$.
        \item[(B)] $\overline{AB}=\overline{DE}$, $\overline{BC}=\overline{EF}$, $\angle CAB = \angle FDE$.
      \end{enumerate}
    \item
      $f$は実数全域で定義された微分可能な実数値関数とする。
      \begin{enumerate}
        \item[(A)] $f'(x)=f(x)$.
        \item[(B)] $f(x)=e^x$.
      \end{enumerate}
    \item 無限数列$\{a_n\}$($n=1,2,3,\dots$)について考える。
      \begin{enumerate}
        \item[(A)] $a_1 = 1$, $a_2=3$, $a_3=5$, $a_4= 7$
        \item[(B)] $\{a_n\}$は等差数列。
      \end{enumerate}
  \end{enumerate}
\end{prob}

\begin{prob}
  $X=\{1,2,3,4,5,6,7,8,9\}$の部分集合$A,B,C$を
  $A=\{1,3,5,7,9\}$, $B=\{1,2,4,5,7,8\}$, $C=\{2,3,5,7\}$とする。
  ベン図を書き、$X$の各要素がどこに含まれるかを書け。
\end{prob}

\begin{prob}
  $A, B \subset \Z$を$A$は偶数全体の集合、$B$は奇数全体の集合とする。
  次の命題を$\forall$, $\exists$, $s.t.$を用いて記号のみの表記に書き換えよ。また、真か偽かを答えよ。
  \begin{enumerate}
    \item 任意の$A$の要素$a$に対して、ある$B$の要素$b$が存在し、$a<b$となる。
    \item ある$A$の要素$a$が存在して、任意の$B$の要素$b$に対して、$a<b$となる。
    \item 任意の$A$の要素$a$に対して、ある$B$の要素$b$が存在し、$a=2b$となる。
    \item 任意の$B$の要素$b$に対して、ある$A$の要素$a$が存在し、$b=a/2$となる。
  \end{enumerate}

\end{prob}

\begin{prob}
  次の条件を$\forall$, $\exists$, $s.t.$, $\Rightarrow$, $\land$, $\lor$などを用いて記号のみで表記せよ。
  ただし、$l,n$は整数、$m$は2以上の整数、$p$は正の整数、$\{A_k\}_{k=1,2,\dots}$はフィボナッチ数列。
  \begin{enumerate}
    \item $n$は偶数。
    \item $n$がフィボナッチ数列に現れる。
    \item $p$は素数。
    \item $l$は$m$を法として$n$と合同。
  \end{enumerate}
\end{prob}


\end{document}
