\documentclass[uplatex]{jsarticle}

\usepackage{amsmath,amsthm,amssymb,amscd}
\usepackage{ascmac}
\usepackage[all]{xy}
\usepackage{enumerate}
\usepackage{graphicx}
\usepackage{mathrsfs}

\theoremstyle{definition}
\numberwithin{equation}{section}
\newtheorem{prob}[section]{問題}

\newcommand{\midbar}{\ \middle| \ }

\newcommand{\mf}[1]{{\mathfrak{#1}}}
\newcommand{\bm}[1]{{\mathbf{#1}}}
\newcommand{\bb}[1]{{\mathbb{#1}}}
\newcommand{\mca}[1]{{\mathcal{#1}}}
\newcommand{\msc}[1]{{\mathscr{#1}}}
\newcommand{\mm}[1]{\mathop{\mathrm{#1}}\nolimits}

\newcommand{\R}{\mathbb{R}}
\newcommand{\C}{\mathbb{C}}
\newcommand{\Z}{\mathbb{Z}}
\newcommand{\N}{\mathbb{N}}
\newcommand{\Q}{\mathbb{Q}}

\newcommand{\Zn}{\mathbb{Z}/n\mathbb{Z}}
\newcommand{\Zp}{\mathbb{Z}/p\mathbb{Z}}

\begin{document}
$\N$は自然数全体のなす集合(0は含まない)、$\Z$は整数全体のなす集合、$\R$は実数全体のなす集合、
$\R_{>0}$を正の実数の集合とする。
\begin{prob}
  写像、単射、全射、全単射の定義を述べよ。また、次の写像$f:\Z\to\Z$が全単射、単射、全射、いずれでもない、そもそも写像になっていない、のどれになっているか答えよ。
  \begin{enumerate}
    \item $f(x)=x+a$. ただし、$a$は整数。
    \item $f(x)=\sqrt{x^2+4}$.
    \item
      \[
        f(x)=\begin{cases}
          n/2 & (n\text{が偶数}) \\
          n & (n\text{が奇数})
        \end{cases}
      \]
    \item
      \[
        f(x)=\begin{cases}
          n^3 & (n\text{が偶数}) \\
          2n & (n\text{が奇数})
        \end{cases}
      \]
  \end{enumerate}
\end{prob}

\begin{prob}
  次の関数$f$と定義域$A$に対し、値域を答えよ。
  \begin{enumerate}
    \item $f(x)=\sin x$, $A=[-\pi/6, 3\pi/4]$.
    \item $f(x)=\log x$, $A=(0, 1]$.
    \item $f(x)=1/x$, $A=(-3,4)\setminus{\{0\}}$.
  \end{enumerate}
\end{prob}

\begin{prob}
  次の関数$f$, $g$の合成$g\circ f$を計算せよ。
  \begin{enumerate}
    \item $f(x)=2x+1$, $g(x)=x^2-1$.
    \item $f(x)=2\log x$, $g(x)=a^x$. ただし、$a>0$とする。
    \item
      \begin{align*}
        f(x)&=\frac{ax+b}{cx+d}\\
        g(x)&=\frac{dx-b}{-cx+a}.
      \end{align*}
      ただし、$ad-bc \not= 0$とする。
  \end{enumerate}

\end{prob}

\begin{prob}
  全単射$f:\N\to\N\times\N$を構成し、それが全単射になっていることを証明せよ。
\end{prob}


\end{document}
