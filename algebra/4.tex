\documentclass[uplatex]{jsarticle}

\usepackage{amsmath,amsthm,amssymb,amscd}
\usepackage{ascmac}
\usepackage[all]{xy}
\usepackage{enumerate}
\usepackage{graphicx}
\usepackage{mathrsfs}

\theoremstyle{definition}
\numberwithin{equation}{section}
\newtheorem{prob}[section]{問題}

\newcommand{\midbar}{\ \middle| \ }

\newcommand{\mf}[1]{{\mathfrak{#1}}}
\newcommand{\bm}[1]{{\mathbf{#1}}}
\newcommand{\bb}[1]{{\mathbb{#1}}}
\newcommand{\mca}[1]{{\mathcal{#1}}}
\newcommand{\msc}[1]{{\mathscr{#1}}}
\newcommand{\mm}[1]{\mathop{\mathrm{#1}}\nolimits}

\newcommand{\R}{\mathbb{R}}
\newcommand{\C}{\mathbb{C}}
\newcommand{\Z}{\mathbb{Z}}
\newcommand{\Q}{\mathbb{Q}}

\newcommand{\Zn}{\mathbb{Z}/n\mathbb{Z}}
\newcommand{\Zp}{\mathbb{Z}/p\mathbb{Z}}

\begin{document}
$\R$を実数体、$F$を体、$V$を$F$ベクトル空間とする。
\begin{prob}
  $v_1,\dots,v_n \in V$が、($F$上)一次独立であることの定義を述べよ。
\end{prob}

\begin{prob}
  $v_1,v_2,v_3,v_4\in \R^4$を
  \[
    v_1 = \begin{pmatrix}
      1 \\ 0 \\ 1 \\ 0
    \end{pmatrix}
    , v_2 = \begin{pmatrix}
      0 \\ 1 \\ 0 \\ 1
    \end{pmatrix}
    , v_3 = \begin{pmatrix}
      -1\\ 1 \\ -1 \\ 1
    \end{pmatrix}
    , v_4 = \begin{pmatrix}
      1 \\ -1 \\ -1 \\ 1
    \end{pmatrix}
  \]
  とする。次のうち、$\R$上一次独立になるベクトルの組をすべて選べ。
  \begin{enumerate}
    \item $v_1,v_2$
    \item $v_1,v_2,v_3$
    \item $v_2,v_3$
    \item $v_1,v_2,v_4$
    \item $v_1,v_2,v_3,v_4$
  \end{enumerate}
\end{prob}

\begin{prob}
  $u=(a,b), v=(c,d)\in \R^2$が$\R$上一次独立である必要十分条件が$ad-bc \not=0$であることを示せ。(ヒント:対偶)
\end{prob}

\begin{prob}
  1と$\sqrt{2}$が$\Q$上一次独立であるとはどういうことか、一次独立の定義に従い述べよ。また、これを示せ。
\end{prob}
\end{document}

