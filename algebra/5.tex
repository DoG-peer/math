\documentclass[uplatex]{jsarticle}

\usepackage{amsmath,amsthm,amssymb,amscd}
\usepackage{ascmac}
\usepackage[all]{xy}
\usepackage{enumerate}
\usepackage{graphicx}
\usepackage{mathrsfs}

\theoremstyle{definition}
\numberwithin{equation}{section}
\newtheorem{prob}[section]{問題}

\newcommand{\midbar}{\ \middle| \ }

\newcommand{\mf}[1]{{\mathfrak{#1}}}
\newcommand{\bm}[1]{{\mathbf{#1}}}
\newcommand{\bb}[1]{{\mathbb{#1}}}
\newcommand{\mca}[1]{{\mathcal{#1}}}
\newcommand{\msc}[1]{{\mathscr{#1}}}
\newcommand{\mm}[1]{\mathop{\mathrm{#1}}\nolimits}

\newcommand{\R}{\mathbb{R}}
\newcommand{\C}{\mathbb{C}}
\newcommand{\Z}{\mathbb{Z}}
\newcommand{\Q}{\mathbb{Q}}

\newcommand{\Zn}{\mathbb{Z}/n\mathbb{Z}}
\newcommand{\Zp}{\mathbb{Z}/p\mathbb{Z}}

\begin{document}
$\R$を実数体、$F$を体、$V$を有限次元$F$ベクトル空間とする。
\begin{prob}
  $v_1,v_2,v_3,v_4\in \R^3$を
  \[
    v_1 = \begin{pmatrix}
      1 \\ 0 \\ -1
    \end{pmatrix}
    , v_2 = \begin{pmatrix}
      2 \\ -1 \\ -1
    \end{pmatrix}
    , v_3 = \begin{pmatrix}
      1 \\ 2 \\ -3
    \end{pmatrix}
    , v_4 = \begin{pmatrix}
      1 \\ 1 \\ 1
    \end{pmatrix}
  \]
  とする。次に挙げるベクトルの組に対して、それが生成する部分空間の次元を答えよ。
  \begin{enumerate}
    \item $v_1,v_2$
    \item $v_1,v_2,v_3$
    \item $v_1,v_2,v_3,v_4$
  \end{enumerate}
\end{prob}

\begin{prob}
  次の$\R^4$の部分空間について、基底を挙げ、それが実際に基底になっていることを示し、その部分空間の次元を答えよ。
  \begin{enumerate}
    \item $\left\{ (x,y,z,w)\in\R^4 \midbar x+y+z+w=0 \right\}$
    \item $\left\{ (x,y,z,w)\in\R^4 \midbar x+y=z+w=0 \right\}$
  \end{enumerate}
\end{prob}

\begin{prob}
  $V_1$,$V_2$を$V$の部分空間とする。
  \begin{enumerate}
    \item $V_1+V_2$の定義を述べよ。
    \item $V_1\cap V_2$が$V$の部分空間になることを示せ。
    \item $V_1+V_2$の次元を$V_1$,$V_2$,$V_1\cap V_2$の次元で書き表せ。(答えだけでよい)
    \item $V_1\cup V_2$は必ずしも$V$の部分空間になるとは限らない。反例を挙げ、反例になっていることを示せ。
  \end{enumerate}
\end{prob}

\begin{prob}
  $n$を非負整数とする。$V_n$を次数$n$以下の$x$の実数係数多項式のなすベクトル空間とする。
  つまり、$V_n=\left\{ a_nx^n+a_{n-1}x^{n-1}+\dots+ a_0\midbar a_0,\dots,a_n\in\R \right\}$とする。
  \begin{enumerate}
    \item $V_n$の次元を求めよ。
    \item $V_n$の部分空間$U_n=\left\{f(x)\in V_n\midbar f(x)\text{は偶関数}\right\}$の次元を求めよ。
    \item $a\in\R$とする。$V_n$の部分空間$W_n=\left\{f(x)\in V_n\midbar f(a)=0\right\}$の次元を求めよ。
  \end{enumerate}
\end{prob}
\end{document}

