\documentclass[uplatex]{jsarticle}

\usepackage{amsmath,amsthm,amssymb,amscd}
\usepackage{ascmac}
\usepackage[all]{xy}
\usepackage{enumerate}
\usepackage{graphicx}
\usepackage{mathrsfs}

\theoremstyle{definition}
\numberwithin{equation}{section}
\newtheorem{prob}[section]{問題}


\newcommand{\mf}[1]{{\mathfrak{#1}}}
\newcommand{\bm}[1]{{\mathbf{#1}}}
\newcommand{\bb}[1]{{\mathbb{#1}}}
\newcommand{\mca}[1]{{\mathcal{#1}}}
\newcommand{\msc}[1]{{\mathscr{#1}}}
\newcommand{\mm}[1]{\mathop{\mathrm{#1}}\nolimits}

\newcommand{\R}{\mathbb{R}}
\newcommand{\C}{\mathbb{C}}
\newcommand{\Z}{\mathbb{Z}}
\newcommand{\Q}{\mathbb{Q}}

\newcommand{\Zn}{\mathbb{Z}/n\mathbb{Z}}

\begin{document}
\begin{prob}
  群、環、体の定義を述べよ。
\end{prob}

\begin{prob}
$n$を正の整数とする。$\Zn= \left\{ 0,1,\dots, n-1 \right\}$とおく。通常の足し算を行った後$n$で割った余りをとるという演算で$\Zn$が群になることを示せ。
\end{prob}

\begin{prob}
  $\Zn$に積を定義し環にせよ。さらに、$n$が素数のとき、体になることを示せ。(積を上手く定義しないと体にはならない。)
\end{prob}

\begin{prob}
  それぞれの場合の$n$に対して、$\Zn$での方程式を解け。
  \begin{enumerate}
    \item $n=2$のとき、$x^2 + 1 = 0$.
    \item $n=3$のとき、$x^2 + 1 = 0$.
    \item $n=17$のとき、$x^4 + 1 = 0$.
    \item $n=7$のとき、$x y = 4$.
    \item $n=8$のとき、$x y = 4$.
  \end{enumerate}
\end{prob}
\end{document}

