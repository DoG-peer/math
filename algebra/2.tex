\documentclass[uplatex]{jsarticle}

\usepackage{amsmath,amsthm,amssymb,amscd}
\usepackage{ascmac}
\usepackage[all]{xy}
\usepackage{enumerate}
\usepackage{graphicx}
\usepackage{mathrsfs}

\theoremstyle{definition}
\numberwithin{equation}{section}
\newtheorem{prob}[section]{問題}

\newcommand{\midbar}{\ \middle| \ }

\newcommand{\mf}[1]{{\mathfrak{#1}}}
\newcommand{\bm}[1]{{\mathbf{#1}}}
\newcommand{\bb}[1]{{\mathbb{#1}}}
\newcommand{\mca}[1]{{\mathcal{#1}}}
\newcommand{\msc}[1]{{\mathscr{#1}}}
\newcommand{\mm}[1]{\mathop{\mathrm{#1}}\nolimits}

\newcommand{\R}{\mathbb{R}}
\newcommand{\C}{\mathbb{C}}
\newcommand{\Z}{\mathbb{Z}}
\newcommand{\Q}{\mathbb{Q}}

\newcommand{\Zn}{\mathbb{Z}/n\mathbb{Z}}

\begin{document}
$R$を可換環、$F$を体とする。
\begin{prob}
  $R$加群、$F$ベクトル空間の定義を述べよ。
\end{prob}

\begin{prob}
  $n$個の$F$の元の列$\left( a_1,\dots, a_n \right)$($a_i\in F$)の集合を$F^n$と書く。和とスカラー倍を次のように定める。
  \begin{align}
\left( a_1,\dots, a_n \right) + \left( b_1,\dots, b_n \right) &= \left(a_1 + b_1,\dots, a_n + b_n \right), \notag \\
  a\left( x_1,\dots, x_n \right) &= \left(ax_1,\dots, ax_n \right) \notag
\end{align}
で定める。
  $F^n$が$F$ベクトル空間であることを示せ。
\end{prob}

\begin{prob}
  \begin{align}
  A &= \left\{ (a_1,\dots, a_n) \in F^n \midbar a_1+\dots+a_n = 0 \right\} \notag \\
  B &= \left\{ (a_1,\dots, a_n) \in F^n \midbar a_1+\dots+a_n = 1 \right\} \notag
  \end{align}
  とおく。
  \begin{enumerate}
    \item 部分ベクトル空間の定義を述べよ。
    \item $A$が$F^n$の部分ベクトル空間であることを示せ。
    \item $B$が$F^n$の部分ベクトル空間でないことを示せ。
  \end{enumerate}
\end{prob}

\begin{prob}
  $\R$は実数体、$\C$は複素数体。

  $\C$が$\R$ベクトル空間になることを示せ。ただし、和は$\C$の和、スカラー倍は、$a\in\R$, $x\in \C$に対しそのスカラー倍を$ax$とすることで定めるものとする。
\end{prob}
\end{document}

